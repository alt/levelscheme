% \iffalse
%<*internal>
%% File: levelscheme.dtx by Arno Trautmann, mail: arno dot trautmann at gmx dot de
\iffalse
%</internal>
%<*readme>
levelscheme tries to offer an easy way of drawing atomic/molecular level schemes in a LaTeX document.

The dtx file has to be compiled with XeLaTeX in order to get the pdf documentation. The package itself can be used with any LaTeX-flavour.

Any comments are welcome! \emph{Please} help me improve this package!

Contact me: arno dot trautmann at gmx dot de.
%</readme>
%<*internal>
\fi
\begingroup
%</internal>
%<*batchfile>
\input docstrip.tex
\keepsilent
\preamble
EXPERIMENTAL CODE

Do not distribute this file without also distributing the
source files specified above.

Do not distribute a modified version of this file under the same name.

\endpreamble
\postamble
Copyright 2010 Arno Trautmann <arno.trautmann@gmx.de>

Distributed under the LaTeX Project Public License,
verson 1.3c or higher (your choice). The latest version of
this license is at: http://www.latex-project.org/lppl.txt

This work is "author-maintained" by Arno Trautmann

This work conists of this file levelscheme.dtx
         and the derived files levelscheme.sty
                                  and  levelscheme.pdf.
\endpostamble
\askforoverwritefalse

\generate{\file{levelscheme.sty}{\from{levelscheme.dtx}{levelscheme}}}
\generate{\file{levelscheme-example.tex}{\from{levelscheme.dtx}{example}}}
%</batchfile>
%<batchfile>\endbatchfile
%<*internal>
\generate{\file{levelscheme.ins}{\from{levelscheme.dtx}{batchfile}}}
\nopostamble\nopreamble
\generate{\file{readme.markdown}{\from{levelscheme.dtx}{readme}}}
\endgroup
%</internal>
%
%<*driver>
\documentclass[a4paper]{ltxdoc}
\usepackage[english]{babel}
\usepackage{
  hyperref,
  xltxtra
}

\hypersetup{%
  pdfborder= 0 0 0,
  colorlinks=true,
  linkcolor= blue,
  pdftitle=levelscheme
}

\def\filedate{2010/06/27}
\def\fileversion{0.a1}

\title{The \textsf{levelscheme} package}
\author{Arno L. Trautmann\thanks{arno.trautmann@gmx.de}}
\date{Version \fileversion\ \filedate}

\EnableCrossrefs
\CodelineIndex
\RecordChanges

%\OnlyDescription
\begin{document}
\maketitle
This is the package |levelscheme| which tries to offer an easy way of drawing atomic/molecular level schemes in a LaTeX document.
 
Any comments are welcome! \emph{Please} help me improve this package!
\tableofcontents
\newpage
\part{Documentation}
\section{Introduction}
This package aims to provide an easy and fast way to draw atomic level schemes. It is designed for theses in the atomic, molecular or quantum optical sciences to help quickly sketch the idea of a certain (experimental) situation. |levelscheme| is \emph{not} intended to offer rigorous calculations and exact potential forms etc. It is just an alternative to drawing things in Paint.

\section{Usage}
There are some simple commands and some that will develop quite elaborate. The most simple way to work with |levelscheme| is to draw levels and connect them by transitions. Every level has a name and can have a label. The transitions can be absorptions, emissions and (not yet implemented) stimulated emissions. Use them by name:
\begin{verbatim}
\level{first}
\level{second}
\absorb{second}{first}
\end{verbatim}
You have to have the process in mind to get the right directions: absorptions are always from the lower level to the upper, but you might as well change the arguments – this will lead to a violation of energy conservation as you get a transition to the lower level \emph{and} absorb a photon …

\subsection{Tuning}
Most of the commands can be tuned to adapt the user’s needs. This is mostly done by a key-value interface, but also some macros are used to set important variables. This all will most probably chang in the next weeks, so no documentation so far.

\StopEventually{}
\clearpage

\part{Implementation: levelscheme.sty}

\DocInput{levelscheme.dtx}
\end{document}
%</driver>
%<*levelscheme>
% \fi
% \setcounter{section}{0}
% \section{Preamble}
% Preamble stuff, package loading. We use |xparse| due to superior macro definition as well as |expl3| (required by xparse) to get the nice |l3keys|.
% 
% Last, special tikz-libraries for arrows etc.
%    \begin{macrocode} 
\ProvidesPackage{levelscheme}
  [2010/06/27 v 0.a1 atomic/molecular level schemes]
\RequirePackage{
  forloop,
  tikz,
  xparse
}
\usetikzlibrary{%
	arrows,%
	decorations.pathmorphing,%
	positioning%
}
\usepgflibrary{%
	shapes.misc%
}
\ExplSyntaxOn
%    \end{macrocode}
% \section{Keys}
% keys:
%    \begin{macrocode}
\keys_define:nn{ls_trans}{
  dir.code:n = \tl_set:Nn\ls_trans_dir{#1},
  right.meta:n = {dir=right},
  left.meta:n = {dir=left},
% kind of arrows used
  arrow.code:n = \tl_set:Nn\ls_trans_arrow{{#1}},
  {<->}.meta:n = \tl_set:Nn\ls_trans_arrow,
%  location of the transition relative to <what?>
  loc.code:n = \tl_set:Nn\ls_trans_loc{#1},
%
  pos.code:n = \tl_set:Nn\ls_trans_pos{#1},
  midway.meta:n = {pos=midway},
  near~end.meta:n = {pos=near~end}
}

\keys_define:nn{ls_pot}{
  domain.code:n = \tl_set:Nn\ls_pot_domain{#1}
}

%% all keys defined, now nice interface:
%% intern command to set keys (misc)
%% this here needs clean up!
\cs_new:Nn\ls_setkeys:n{%
  \keys_set:nn{ls}{#1}%
}%

%% intern command to set keys (transitions only)
\cs_new:Nn\ls_setkeys_trans:n{%
  \keys_set:nn{ls_trans}{#1}
}

%% user level wrapper, only for misc
\NewDocumentCommand\lssetkeys{o m}{%
  \IfNoValueTF{#1}
    {\ls_setkeys:n{#2}}
    {\keys_set:nn{ls_#1}{#2}}
}
%    \end{macrocode}
%\section{Constants and Variables}
% setting up constants/varibles.
%    \begin{macrocode}
\tl_set:Nn\ls_levellength{2}
\tl_set:Nn\ls_leveldistance{1}

\DeclareExpandableDocumentCommand\photonlength{}{2.5}
\DeclareExpandableDocumentCommand\finestructlength{}{2}

\int_new:c{levelnr}
\int_new:c{transdist}

\tl_gset:Nn\c_ls_right{right}

\newcounter{sublevel}  %% here for the for-loop anywhere, as that package does not get clear with expl3-counts
\newcounter{HOlevel}   %% here for the for-loop in the harmonic oszillator

\int_gset:cn{levelnr}{1}
\int_gset:cn{transdist}{1}
%    \end{macrocode}
% User level macros to set them
%    \begin{macrocode}
\DeclareDocumentCommand\leveldistance{m}{
  \tl_gset:Nn\ls_leveldistance{#1}
}

\DeclareDocumentCommand\sublevellength{}{2}
\cs_new:Nn\ls_subleveldistance:{1}

\DeclareDocumentCommand\subleveldistance{m}{
  \cs_set:Nn\ls_subleveldistance:{#1}
}
\ExplSyntaxOff
%    \end{macrocode}
% \section{Level Scheme}
% This is the main environment of this package, and almost everything elese is based on this. Mostly, it is just a wrapper for a |tikzpicture| with some settings.
%    \begin{macrocode}
\ExplSyntaxNamesOn
\DeclareDocumentEnvironment{levelscheme}{O{}}
  {\begin{tikzpicture}[#1,inner sep=0mm,>=stealth]%
    \node (lastlevel) at (0,0){};%
  }
{\end{tikzpicture}}

\DeclareDocumentCommand\level{O{}m}{
\begin{scope}[node distance = \ls_leveldistance and \ls_levellength]
  \node (#2) [below=of lastlevel] {\llap{#1}};
  \node (#2-2) [right=of #2] {};
  \node (#2-3) [right=of #2-2] {};
  \draw (#2.center) -- (#2-3.center);
  \node (lastlevel) at (#2) {};
  \node (lastlevel-3) at (#2-3){};
  \int_set:cn{transdist}{1}  %% return to normal distance
\end{scope}
}

\DeclareDocumentCommand\dashedlevel{O{}m}{
\begin{scope}[node distance = \ls_leveldistance and \ls_levellength]
  \node (#2) [below=of lastlevel] {\llap{#1}};
  \node (#2-2) [right=of #2] {};
  \node (#2-3) [right=of #2-2] {};
  \draw[style=dashed] (#2.center) -- (#2-3.center);
  \node (lastlevel) at (#2) {};
  \node (lastlevel-3) at (#2-3){};
  \int_set:cn{transdist}{1}  %% return to normal distance
\end{scope}
}

%    \end{macrocode}
% A number of sublevels of given distances
%    \begin{macrocode}
\DeclareDocumentCommand\sublevels{O{lastlevel-3}mm}
{
  \node (sublevelstart) [right=of #1.center]{};
  \foreach \x in {#3}{%
    \node (#2-\x) [%
      node distance= \x*\ls_subleveldistance:,%
      above=of sublevelstart.south%
      ]{};
    \draw (#1.center) -- (#2-\x.center) -- +(2,0) node (end-#2-\x) {\hbox to 2em{\hfill \x}};
  }
}
%    \end{macrocode}
% A transition. This is kind of the basic macro for nearly everything interesting. I hope it is configurable enough to be really useful. However, key-value is as expandable as one wishes, so the first argument is a key-value list for everyeverything.
%    \begin{macrocode}
\DeclareDocumentCommand\transition{sO{} mmm}{
  \ls_setkeys_trans:n{loc=\int_use:c{transdist},midway,arrow={->}}  %% bit of a strange but necessary way to define a default
  \ls_setkeys_trans:n{#2}
%
%% construct nodes and draw the thing, define a center at which the photon ends
%
  \node (translow) [node distance = \ls_trans_loc ,right=of #3] {};
  \node (transup) [node distance = \ls_trans_loc,right=of #4] {};
  \draw [\ls_trans_arrow] (translow.center) -- (transup.center) node (photonend)[\ls_trans_pos]{};
%
%% calculate the beginning of the photon, depending wether it comes/goes to/from left or right
%
  \IfBooleanTF{#1}{}{ %% is there a photon in this transition? i.e. no star given
  \tl_if_eq:NNTF\ls_trans_dir\c_ls_right  %% does the photon come from right or left?
      {\node [node distance = \photonlength](photonstart) [right=of photonend] {};}
      {\node [node distance = \photonlength](photonstart) [left=of photonend]   {};}
    \photon[#5]{photonstart}{photonend};
  }
%
  \int_incr:c{transdist}  %% and shift the next transition so they will not overlap. Has to be adjusted for left/right
}
%    \end{macrocode}
% a set of sublevels at constant distances
% \begin{macro}{finestruct}
%    \begin{macrocode}
\NewDocumentCommand\finestruct{mmm}{
  \int_set:cn{transdist}{0}
  \node (finestart) [right=of lastlevel-3]{};
  \forLoop{1}{#1}{sublevel}{
    \node (#3-\thesublevel) [%
      node distance= #2*(-#1/2+(\thesublevel-.5)),%
      above=of finestart.south%
      ]{};
  \draw (lastlevel-3.center) to (#3-\thesublevel.center) -- +(2,0) node (end-#3-\thesublevel) {};
  }
}
%    \end{macrocode}
% \end{macro}
% for shifting the transitions by hand
%    \begin{macrocode}
\NewDocumentCommand\shifttrans{m}{
  \int_add:cn{transdist}{#1}
}
%    \end{macrocode}
% a transition with arrows at both sides to indicate that a cyled transition is meant
%    \begin{macrocode}
\NewDocumentCommand\cycle{O{}mm}{
  \transition*[#1,arrow={<->}]{#2}{#3}{}
}
%    \end{macrocode}
% a single photon from node 1 to node 2
%    \begin{macrocode}
\NewDocumentCommand\photon{O{->}mm}{
  \draw [#1,decorate,decoration={snake,pre length=6pt,post length=2pt},cross out, draw=black] (#2) [cross out, draw=black]-- (#3);
}
%    \end{macrocode}
% absorbtion of a photon, a special transition
%    \begin{macrocode}
\NewDocumentCommand\absorb{O{midway,dir=left}mm}{
  \transition[#1]{#2}{#3}{->}
}
%    \end{macrocode}
% emission of a photon, another special transition
%    \begin{macrocode}
\NewDocumentCommand\emit{O{midway,dir=right}mm}{
  \transition[#1]{#2}{#3}{<-}
}
%    \end{macrocode}
% induced emission as needed for a laser. Not yet implemented.
%    \begin{macrocode}
\NewDocumentCommand\indemit{O{midway,dir=right}mm}{
  \transition[#1]{#2}{#3}{<-}
  \photon{1,1}{2,2}
%% TODO
}
%    \end{macrocode}
% photonless transition
%    \begin{macrocode}
\NewDocumentCommand\relaxation{mm}{
%% TODO
}
%    \end{macrocode}
% \section{$\Lambda$ scheme}
% The environment |lambdascheme| is usefull to draw a $\Lambda$ shaped niveau system.
%    \begin{macrocode}
\NewDocumentEnvironment{lambdascheme}{}
  {\begin{levelscheme}
    \tl_set:Nn\ls_levellength{1}
    \tl_set:Nn\ls_lambdaApos{1.5,4}
    \tl_set:Nn\ls_lambdaBpos{3,1}
    \tl_set:Nn\ls_lambdaCpos{0,0}
    \node (lastlevel) at (\ls_lambdaApos){};
    \level{lambda-A}
    \node (lastlevel) at (\ls_lambdaBpos){};
    \level{lambda-B}
    \node (lastlevel) at (\ls_lambdaCpos){};
    \level{lambda-C}
    \cycle{lambda-C}{lambda-A}
    \shifttrans{-\ls_levellength}
    \cycle{lambda-B}{lambda-A}
  }
  {\end{levelscheme}}
%    \end{macrocode}
% \section{Potentials}
%The macros presented in this section enable you to draw some of the typical potentials needed in atomic and molecular physics. Plans are to implement a hormonic potential, Morse potential, Lennard-Jones potential, a gaussian potential (optical dipole trap) and a linear potential. This is \emph{not} intended for rigorous calculations and scientific things, just for sketching. If you need exakt forms of potentials and level distances, go for a professional program.
%\subsection{MOT}
%    \begin{macrocode}
\NewDocumentCommand\MOT{}{
\begin{levelscheme}
  \level{MOT-upper}
  \tl_set:Nn\ls_motdistance{0.95}
  \node (MOT-plus-right) [node distance=\ls_motdistance,above = of MOT-upper-3] {};
  \node (MOT-plus-left) [node distance=\ls_motdistance,below = of MOT-upper] {};
  \node (MOT-minus-right) [node distance=\ls_motdistance,below = of MOT-upper-3] {};
  \node (MOT-minus-left) [node distance=\ls_motdistance,above = of MOT-upper] {};

  \leveldistance{0.5}
  \dashedlevel{MOT-middle}
  \leveldistance{3}
  \level{MOT-lower}

  \draw (MOT-plus-left) -- (MOT-plus-right);
  \draw (MOT-minus-left) -- (MOT-minus-right);
  \absorb[loc=0.3*\ls_levellength]{MOT-lower}{MOT-middle}
  \absorb[right,loc=1.7*\ls_levellength]{MOT-lower}{MOT-upper}
\end{levelscheme}
}
%    \end{macrocode}
% \subsection{Harmonic Oszillator}
%    \begin{macrocode}
\tl_set:Nn\lsHOhight{0.5}
\tl_set:Nn\lsHOwidth{0.5}

\NewDocumentEnvironment{EIT}{}
{\begin{levelscheme}
\
\tl_set:Nn\ls_levellength{0.2}
\level{mynewlevelshit}
}
{\end{levelscheme}}

\group_begin:
  \char_set_lccode:nn { `\! } { `\: }
  \char_make_other:N \!
\tl_to_lowercase:n {
  \group_end:
  \cs_new:Npn\setdomains#1#2{
  \pgfkeys {tikz/domain = #1 ! #2}
  }
}

\NewDocumentCommand\HO{O{4}m}{
\setdomains{-2}{2}
  \begin{tikzpicture}
    \draw plot (\x,{\x^2*\lsHOwidth}){};
    \forLoop{0}{#1}{HOlevel}{
      \def\HOL{{(\theHOlevel+0.5)*\lsHOhight}}
      \draw ({sqrt(\HOL/\lsHOwidth)},\HOL) -- ({-sqrt(\HOL/\lsHOwidth)},\HOL){};
  }
  \end{tikzpicture}
}
%    \end{macrocode}
%\subsection{Lennard-Jones potential}
%    \begin{macrocode}
  \DeclareDocumentCommand\potlevel{m}{
    \def\potlevelleft{%
      {2^(1/6)/\ljSigma*((sqrt(\ljEpsilon^2+#1*\ljEpsilon)/#1) - (\ljEpsilon/#1))^(1/6)-0.03*#1}% %% WTF correction summand needed?!?
    }
    \def\potlevelright{
      {2^(1/6)/\ljSigma*((-sqrt(\ljEpsilon^2+#1*\ljEpsilon)/#1) - (\ljEpsilon/#1))^(1/6)}%
    }
    \draw (\potlevelleft,#1) -- (\potlevelright,#1){};
}
\NewDocumentEnvironment{lennardjones}{mm}
{
  \gdef\ljEpsilon{#1}
  \gdef\ljSigma{#2}
  \setdomains{1}{4}
  \begin{tikzpicture}
  \draw plot[smooth] (\x,{4*\ljEpsilon*((\ljSigma*\x)^(-12)-(\ljSigma*\x)^(-6))}){};
}{
  \end{tikzpicture}
}
\ExplSyntaxNamesOff
%    \end{macrocode}
%\subsection{Morse potential}
%    \begin{macrocode}
\NewDocumentEnvironment{morsepotential}{O{}}
{
%%%  \ls_setkeys_potential{#1}
  \begin{tikzpicture}[domain=1:3]
    \draw plot[smooth] (\x,{(1*(1-exp(-0.7*(\x*2-1))))^2}){};
    \draw plot[smooth] (\x,{(1*(1-exp(-0.7*(\x*2-1))))^2+0.3}){};
}{
  \end{tikzpicture}
}

%    \end{macrocode}
% \section{Known Bugs}
% This should be a list of (serious) bugs. Please report any of them to me!
% \begin{itemize}
% \item MOT beams should be aligned and all at same hight
% \item offset in Lennard-Jones potential has to be removed. No idea what is going wrong there
% \end{itemize}
% \newpage\appendix
% \section*{todo – missing features}
% Here a section with some ideas that could be implemented.
% \begin{itemize}
% \item Feshbach resonances
% \end{itemize}
% \iffalse
%</levelscheme>
% \fi
% \Finale
\endinput